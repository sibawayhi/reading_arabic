\documentclass[11pt]{article}



% \usepackage[xetex,svgnames,x11names]{xcolor}
%%%%%%%%%%%%%%%%  START XETEX SETUP  %%%%%%%%%%%%%%%%
\usepackage{fontspec}
%%\usepackage{xunicode}
\usepackage{xltxtra}
% \defaultfontfeatures{Mapping=tex-text}

%\setmainfont{Cochin}
\setmainfont{Gentium Book Plus}

% \newfontfamily\symfam{Apple Symbols}
%%\newfontfamily\symfam2{Arial Unicode MS}
%%\newfontfamily\symfam2{/System/Library/Fonts/ZapfDingbats.ttf}
%%\newfontfamily\symfam2{Menlo}

% \newfontfamily\arfont{Scheherazade:script=arab}
% \newfontfamily\arsegfont[Scale=1.2]{Scheherazade:script=arab}
% \newfontfamily\arversefont[Scale=1.5]{Scheherazade:script=arab}

\newfontfamily\englishfontit{Gentium Book Plus Italic}
\newfontfamily\arabicfont[Scale=1.6]{Scheherazade:script=arab}

% \newfontfamily\artitle[Scale=10.0]{Scheherazade:script=arab}
% \newfontfamily\arsubtitle[Scale=3.0]{Scheherazade:script=arab}
% \newfontfamily\arhwfont[Scale=1]{KufiStandardGK:script=arab}
% \newfontfamily\enhwfont[Scale=1.2]{Lucida Grande}

% \newfontfamily\arexfont[Scale=1.2]{Scheherazade:script=arab}

% \newcommand{\exar}[1]{\RLE{\arexfont #1}}

% \newcommand{\ar}[1]{{\beginR\arfont #1\endR}}
% \newcommand{\arnote}[1]{{\RL{\normalsize\arfont #1}}}
% \newcommand{\arheadword}[1]{\beginR{\arhwfont #1}\endR}
% \newcommand{\arx}[2]{\xlit{\itshape #1} ({\beginR\arfont #2\endR})}

% \newcommand{\en}[1]{{\beginL\sffamily #1\endL}}
% \newcommand{\enheadword}[1]{\beginL{\enhwfont #1}\endL}
% \newcommand{\enx}[1]{{\xlit{\itshape #1}}}

% \newfontfamily\entext[Scale=.8]{Geneva}

% \newfontfamily\xlit{Charis SIL}

% \newfontfamily\arxen{Charis SIL}

% % \newfontfamily\roman{Gentium Book Plus}
% \newfontfamily\italic{Gentium Book Plus Italic}
% \newfontfamily\ipafamily{Doulos SIL}
% \newfontfamily\ital{Andale Mono}
% \newfontfamily\symbol[Scale=.8]{Apple Symbols}
\let\XeTeX\undefined
\let\XeLaTeX\undefined
%%%%%%%%%%%%%%%%  END XETEX SETUP  %%%%%%%%%%%%%%%%
\usepackage[pdfusetitle]{hyperref} % Creates hyperlinks and index in the PDF document, preferably load after biblatex
\hypersetup{
    bookmarks=true,         % show bookmarks bar?
    unicode=true,          % non-Latin characters in Acrobat’s bookmarks
    pdftoolbar=true,        % show Acrobat’s toolbar?
    pdfmenubar=true,        % show Acrobat’s menu?
    pdffitwindow=false,     % window fit to page when opened
    pdfstartview={FitH},    % fits the width of the page to the window
    pdftitle={My title},    % title
    pdfauthor={Author},     % author
    pdfsubject={Subject},   % subject of the document
    pdfcreator={Creator},   % creator of the document
    pdfproducer={Producer}, % producer of the document
    pdfkeywords={keyword1} {key2} {key3}, % list of keywords
    pdfnewwindow=true,      % links in new window
    colorlinks=true,        % false: boxed links; true: colored links
    linkcolor=blue,         % color of internal links
    citecolor=green,        % color of links to bibliography
    filecolor=magenta,      % color of file links
    urlcolor=cyan           % color of external links
}

\usepackage{polyglossia}
\setmainlanguage{english}
\setotherlanguage{arabic}

%% \usepackage[extrafootnotefeatures]{bidi}  %% for xetex, load last

\newcommand{\sib}{Sībawayhi}
\newcommand{\ism}{\textit{ism}}
\newcommand{\harf}{\textit{ḥarf}}
\newcommand{\huruf}{\textit{ḥurūf}}
\newcommand{\fil}{\textit{fiʿl}}
\newcommand{\maana}{\textit{maʕnā}}
\newcommand{\damma}{\textit{ḍamma}}
\newcommand{\fatha}{\textit{fatḥa}}
\newcommand{\kasra}{\textit{kasra}}
\newcommand{\sukun}{\textit{sukūn}}
\newcommand{\rafu}{\textit{rafʿ}}
\newcommand{\nasb}{\textit{naṣb}}
\newcommand{\jarr}{\textit{jarr}}
\newcommand{\waqf}{\textit{waqf}}
\newcommand{\tanwin}{\textit{tanwīn}}
\newcommand{\hasan}{\textit{ḥasan}}
\newcommand{\qabih}{\textit{qabīḥ}}
\newcommand{\kitab}{\textit{Kitāb}}
\newcommand{\xabar}{\textit{xabar}}
\newcommand{\maful}{\textit{mafʿūl}}
\newcommand{\zarf}{\textit{ẓarf}}
\newcommand{\qawl}{\textit{qawl}}
\newcommand{\aqwal}{\textit{aqwāl}}
\newcommand{\sama}{\textit{samāʕ}}
\newcommand{\qiyas}{\textit{qiyās}}
\newcommand{\illa}{\textit{ʕilla}}
\newcommand{\taqdir}{\textit{taqdīr}}
\newcommand{\amal}{\textit{ʕamal}}
\newcommand{\hadathan}{\textit{ḥadaθān}}
\newcommand{\hal}{\textit{ḥāl}}
\newcommand{\asl}{\textit{aṣl}}
\newcommand{\almutakallim}{\textit{al-mutakallim}}
\newcommand{\almuxatab}{\textit{al-muxātab}}
\newcommand{\kalam}{\textit{kalām}}
\newcommand{\nahwiyyun}{\textit{naḥwiyyūn}}
%\newcommand{}{\textit{}}


%% ʿ

%%%%%%%%%%%%%%%%%%%%%%%%%%%%%%%%%%%%%%%%%%%%%%%%%%%%%%%%%%%%%%%%
%%%%%%%%%%%%%%%%%%%%%%%%%%%%%%%%%%%%%%%%%%%%%%%%%%%%%%%%%%%%%%%%
\begin{document}

\title{Reading Arabic}
\author{G. A. Reynolds}
\date{Today}
\maketitle

\begin{abstract}
A Tutorial and Guide to Arabic Orthography
\end{abstract}

\tableofcontents

\section{Overview: \textit{ḥarf}, \textit{ḥarakah}, and \textit{taʃkīl}}

Arabic \textit{ḥurûf} have two or four forms, but fortunately each has
a kernel that is more or less the same in each form.

\begin{itemize}
\item freestanding, initial, medial, and final; e.g. \textarabic{ت ، تـ
  ، ـتـ ، ـت}
\item freestanding and final. These have roughly similar shapes. The
  final forms are identical to the freestanding forms, except they are
  tied to the previous form by a ligature:
  \textarabic{ا ـا ، د ـد ، ذ ـذ ، ر ـر ، ز ـز ، و ـو}
\end{itemize}

Note that these forms are determined by the preceding and following
letters rather than position in a word. For example, a word may be
composed entirely of free-standing forms, as in \textarabic{دَوَّرَ}
``dawwara'' or \textarabic{وَرْد} ``ward''; entirely of connected forms,
as in \textarabic{كتب ، منطلق}, or a combination of both, as in
\textarabic{موضع ، كتاب ، دخل ن ضاربون}

Groups:

\begin{itemize}
\item toothed forms: \textarabic{ب ، ت ، ث ، ي}
\item three forms without or with a dot, above or below: \textarabic{ج~،~ح~،~خ}
\item paired forms, without and with a diacritic:
  \textarabic{(د ، ذ) (ر ، ز) (س~،~ش) (ص ، ض) (ط ، ظ) (ع ، غ) (ف ، ق)}
\item remaining forms: \textarabic{ا ، ك ، ل ، م ، و}
\end{itemize}

\section{Letters and Diacritics}

The orthographic “letters” of Arabic are called variously:

\begin{itemize}
\item \textarabic{حُرُوفُ الْهَجَاءِ}
\item \textarabic{حُرُوفُ التَّهَجَِي}
\item \textarabic{حُرُوفُ الْمُعْجَمِ}
\item \textarabic{الْحُرُوفُ الْمُقَطَّعَةُ}
\end{itemize}

Per Lane: \textarabic{حُرُوفُ الْمُعْجَمِ} an appellation of \textit{The
  letters of the alphabet [of the language of the Arabs]}, most of
which are distinguished by being dotted from the letters of other
peoples, means \textarabic{حُرُوفُ الْجَطِّ الْمُعْجَمِ}, \textit{the letters of
  the dotted character}

This may be a reference to the \textarabic{ض}, which is distinctive of
the Arabic language, which is therefore sometimes called
\textarabic{لُغَةُ الضَّادِ} \textit{the language of the ḍād}.

%%%%%%%%%%%%%%%%%%%%%%%%%%%%%%%%
\subsection{Orderings}

\paragraph{Traditional: Abjadiyyah}

The \href{https://en.wikipedia.org/wiki/Abjad}{Abjadiyyah} order is
the traditional order:

\textarabic{
ا\,ب\,ج\,د\,ه\,و\,ز\,ح\,ط\,ي\,ك\,ل\,م\,ن\,س\,ع\,ف\,ص\,ق\,ر\,ش\,ت\,ث\,خ\,ذ\,ض\,ظ\,غ}

\vspace{10pt}
\noindent
This is usually partitioned into pronouncable “phrases”:

\vspace{8pt}
\textarabic{أَبْجَدْ هَ‍وَّزْ حَطِّيْ كَلَمَنْ سَعْفَصْ قَرَشَتْ ثَخَذْ ضَظَغْ}

\paragraph{Modern: Alif-baa'iyah}

The traditional ordering has been replaced by an ordering based on
shapes and diacritics, which we use in this document:

\vspace{10pt}

\textarabic{ا\,ب\,ت\,ث\,د\,ذ\,ر\,ز\,ش\,س\,ص\,ض\,ط\,ظ\,ع\,غ\,ف\,ق\,ل\,م\,ن\,ه\,و\,ي}

%%%%%%%%%%%%%%%%%%%%%%%%%%%%%%%%
\subsection{The \textit{Ḥurûf} (\textarabic{الْحُرُوْف})}

Forms are listed in right-to-left order: freestanding, initial,
medial, final.

Transliterations are for this document only; they deviate somewhat
from standard transliteration schemes.

%%%%%%%%%%%%%%%%%%%%%%%%%%%%%%%%
\subsubsection{ \textarabic{ا}  \textit{ā} as in b\textit{ā}t}
This is the most complex ḥarf. See below.

\noindent Name: alif

\noindent Forms: \textarabic{ا ، ـا}

\paragraph{\textarabic{ٱ} alif al-wasl}

note on spelling when alif-lam prefixes alif wasl, e.g.
\textarabic{الابن}. The alif-lam is maksur, \textarabic{الِ}, followed
by the “null” alif wasl, without tashkeel. Standard orthography
“folds” the lam and following alif, like so: \textarabic{لا}, yielding
\textarabic{الِابْنُ}. Spread out for clearity: \textarabic{الِـابْنُ}. The
critical point is that this alif, between the lam and the
\textarabic{ب}, is “null” - it has no phonetic role, so an equivalent
phonetic spelling would be \textarabic{الِبْنُ}. But this would introduce
ambiguity, since without the voweling, \textarabic{البن} could be read
as \textarabic{الْبُنُّ}. The leading alif wasl serves to disambiguate,
indicating the word boundary.

\paragraph{\textarabic{آ} alif al-madd}
\paragraph{\textarabic{ـٰ}\  dagger alif}
\paragraph{\textarabic{ى} alif maqsurah}

The strangest beast in the menagerie. The alif maqsura may be written
as either \textarabic{ى}, as in \textarabic{فتى}, or \textarabic{ا} as
in \textarabic{غَزَا}. Which demonstrates that \textit{alif} is a
concept, rather than a mere letter.

See also \textarabic{الممدود} e.g. \textarabic{بِنَاْء}.

WARNING: in printed text, final \textarabic{ي} is often written
without dots, so it looks like a proper alif maqsurah.



%%%%%%%%%%%%%%%%%%%%%%%%%%%%%%%%
\subsubsection{ \textarabic{ب}  \textit{b} as in \textit{b}ottom}
Mnemonic: \textit{b}ottom dot

\noindent Name: baa'

\noindent Forms: \textarabic{ب ، بـ ، ـبـ ، ـب}


%%%%%%%%%%%%%%%%%%%%%%%%%%%%%%%%
\subsubsection{ \textarabic{ت} \textit{t} as in \textit{t}wo}
Mnemonic: \textit{t}wo dots on a tooth.

\noindent Name: taa'

\noindent forms: \textarabic{ت ، تـ ، ـتـ ، ـت}

\subsubsection{ \textarabic{ث} \textit{th} as in \textit{th}ree}
Mnemonic:  \textit{th}ree dots on a tooth.

\noindent Name: taa'

\noindent Transliteration: θ (Greek small letter theta)

\noindent Forms: \textarabic{ث ، ثـ ، ـثـ ، ـث}

\subsubsection{ \textarabic{ج} \textit{j} as in \textit{j}oker}
Mnemonic:  ?

\noindent Name: jīm (jeem)

\noindent Forms: \textarabic{ج ، جـ ، ـجـ ، ـج}

\subsubsection{ \textarabic{ح} \textit{ḥ} as in \textit{ḥ}arsh}
Pronunciation: like the ``soft'' h \textarabic{ه}, but much harsher.
See section X below.

\noindent Name: ḥaa

\noindent Transliteration: ḥ (h with dot below)

\noindent Mnemonic:  ?

\noindent Forms: \textarabic{ح ، حـ ، ـحـ ، ـح}

%%%%%%%%%%%%%%%%%%%%%%%%%%%%%%%%
\subsubsection{ \textarabic{خ} \textit{kh} as in lo\textit{kh} ness}
Pronunciation: like ``\textit{ch}'' in Scottish ``lo\textit{ch}'' or
German ``a\textit{ch}'' but stronger; see below.

\noindent Name: xaa'

\noindent Transliteration: x - ``kh'' is unsuitable, since both
\textit{k} and \textit{h} are used, for \textarabic{ك} and
textarabic{ه} respectively.

\noindent Mnemonic:  ?

\noindent Forms: \textarabic{خ ، خـ ، ـخـ ، ـخ}

%%%%%%%%%%%%%%%%%%%%%%%%%%%%%%%%
\subsubsection{ \textarabic{د} \textit{d} as in \textit{d}og}
Mnemonic:  ?

This is our first letter with only two forms, freestanding and final.

\noindent Name: dal

\noindent Forms: \textarabic{د ، ـد}

\subsubsection{ \textarabic{ذ} \textit{th} as in mo\textit{th}er, fa\textit{th}er}
Mnemonic:  a dog with a dot is ...?

\noindent Forms: \textarabic{ذ ، ـذ}

%%%%%%%%%%%%%%%%%%%%%%%%%%%%%%%%
\subsubsection{ \textarabic{ر} \textit{r} as in \textit{r}ada\textit{r}}
Mnemonic:  ?

\noindent Name: raa'

\noindent Forms: \textarabic{ر ، ـر}

%%%%%%%%%%%%%%%%%%%%%%%%%%%%%%%%
\subsubsection{ \textarabic{ز} \textit{z} as in \textit{z}ebra}
Mnemonic:  ?

\noindent Name: zayn

\noindent Forms: \textarabic{ز ، ـز}

%%%%%%%%%%%%%%%%%%%%%%%%%%%%%%%%
\subsubsection{ \textarabic{س} ``thin'' \textit{s} as in \textit{s}ee}
Mnemonic:  ?

\noindent Name: sīn (seen)

\noindent Forms: \textarabic{س ، سـ ، ـسـ ، ـس}

%%%%%%%%%%%%%%%%%%%%%%%%%%%%%%%%
\subsubsection{ \textarabic{ش} \textit{sh} as in \textit{sh}ow}
Mnemonic:  three dots on three teeth is \textit{sh}owing off

\noindent Name: shīn (sheen)

\noindent Transliteration: ʃ (latin small letter esh)

\noindent Forms: \textarabic{ش ، شـ ، ـشـ ، ـش}

%%%%%%%%%%%%%%%%%%%%%%%%%%%%%%%%
\subsubsection{ \textarabic{ص} ``thick'' \textit{ṣ}}
Pronunciation: see below.

\noindent Name: ṣaud

\noindent Transliteration: ṣ

\noindent Mnemonic:  ?

\noindent Forms: \textarabic{ص ، صـ ، ـصـ ، ـص}

%%%%%%%%%%%%%%%%%%%%%%%%%%%%%%%%
\subsubsection{ \textarabic{ض} ``thick'' \textit{ḍ}}
Pronunciation: similar to \textarabic{د}; see below.

\noindent Name: ḍaud

\noindent Transliteration: ḍ (d with dot below)

\noindent Mnemonic:  ?

\noindent Forms: \textarabic{ض ، ضـ ، ـضـ ، ـض}

%%%%%%%%%%%%%%%%%%%%%%%%%%%%%%%%
\subsubsection{ \textarabic{ط} ``thick'' \textit{ṭ}}
Pronunciation: similar to \textarabic{د}; see below.

\noindent Name: ṭaa

\noindent Transliteration: ṭ (t with dot below)

\noindent Mnemonic:  ?

\noindent Forms: \textarabic{ط ، طـ ، ـطـ ، ـط}

%%%%%%%%%%%%%%%%%%%%%%%%%%%%%%%%
\subsubsection{ \textarabic{ظ} ``thick'' \textit{ẓ}}
Pronunciation: similar to \textarabic{ذ}, \textarabic{ز}; see below.

\noindent Name: ẓaa

\noindent Transliteration: ẓ (z with dot below)

\noindent Mnemonic:  ?

\noindent Forms: \textarabic{ظ ، ظـ ، ـظـ ، ـظ}

%%%%%%%%%%%%%%%%%%%%%%%%%%%%%%%%
\subsubsection{ \textarabic{ع} (unpronouncable ;)}
Pronunciation: see below

\noindent Name: 'ayn

\noindent Transliteration: 9 (rhymes with ayn, vaguely similar shape)

\noindent Mnemonic:  ?

\noindent Forms: \textarabic{ع ، عـ ، ـعـ ، ـع}

%%%%%%%%%%%%%%%%%%%%%%%%%%%%%%%%
\subsubsection{ \textarabic{غ} \textit{g} as in \textit{g}ar\textit{g}le}
Pronunciation: a guttural, rolling, hard /g/ sound

\noindent Name: ghayn

\noindent Transliteration: g

\noindent Mnemonic:  ?

\noindent Forms: \textarabic{غ ، غـ ، ـغـ ، ـغ}

%%%%%%%%%%%%%%%%%%%%%%%%%%%%%%%%
\subsubsection{ \textarabic{ف} \textit{f} as in \textit{f}face}

\noindent Name: faa'

\noindent Transliteration: f

\noindent Mnemonic:  ?

\noindent Forms: \textarabic{ف ، فـ ، ـفـ ، ـف}

%%%%%%%%%%%%%%%%%%%%%%%%%%%%%%%%
\subsubsection{ \textarabic{ق} \textit{q} as in \textit{q}ueue}
Pronunciation: like /k/ but farther back in the throat; see below

\noindent Name: qaf

\noindent Transliteration: q

\noindent Mnemonic:  ?

\noindent Forms: \textarabic{ق ، قـ ، ـقـ ، ـق}

%%%%%%%%%%%%%%%%%%%%%%%%%%%%%%%%
\subsubsection{ \textarabic{ك} \textit{k} as in \textit{k}ick}

\noindent Name: kaf

\noindent Transliteration: k

\noindent Mnemonic:  ?

\noindent Forms: \textarabic{ك ، كـ ، ـكـ ، ـك}

%%%%%%%%%%%%%%%%%%%%%%%%%%%%%%%%
\subsubsection{ \textarabic{ل} \textit{l} as in \textit{l}amb}

\noindent Name: lam

\noindent Transliteration: l

\noindent Mnemonic:  ?

\noindent Forms: \textarabic{ل ، لـ ، ـلـ ، ـل}

%%%%%%%%%%%%%%%%%%%%%%%%%%%%%%%%
\subsubsection{ \textarabic{م} \textit{m} as in \textit{m}i\textit{m}e}

\noindent Name: mīm (meem)

\noindent Transliteration: m

\noindent Mnemonic:  ?

\noindent Forms: \textarabic{م ، مـ ، ـمـ ، ـم}

%%%%%%%%%%%%%%%%%%%%%%%%%%%%%%%%
\subsubsection{ \textarabic{ن}  \textit{n} as in \textit{n}oo\textit{n}, o\textit{n}e}

\noindent Name: nûn (noon)

Mnemonic:  o\textit{n}e dot on a tooth.

NB: this is a “tooth” letter, like \textarabic{ب ، ت ، ي}, but the
freestanding and final forms are a little deeper.

\noindent Forms: \textarabic{ن ، نـ ، ـنـ ، ـن}

%%%%%%%%%%%%%%%%%%%%%%%%%%%%%%%%
\subsubsection{ \textarabic{ه} soft \textit{h} as in  \textit{h}um, \textit{h}am, \textit{h}im }

\noindent Name: haa'

\noindent Transliteration: h

\noindent Mnemonic:  ?

\noindent Forms: \textarabic{ه ، هـ ، ـهـ ، ـه}

%%%%%%%%%%%%%%%%%%%%%%%%%%%%%%%%
\subsubsection{ \textarabic{و} \textit{w} as in  \textit{w}ater,
  \textit{oo} as in b\textit{oo}t}

\noindent Name: waw

\noindent Transliteration: w

\noindent Mnemonic:  ?

\noindent Forms: \textarabic{و ، ـو}

%%%%%%%%%%%%%%%%%%%%%%%%%%%%%%%%
\subsubsection{\textarabic{ي}  \textit{y} as in \textit{y}et;
  \textit{ee} as in f\textit{ee}t; \textit{y} as in fish\textit{y}}
Mnemonic:  ?

\noindent Name: yaa'

\noindent Transliteration: y

\noindent Mnemonic:  ?

\noindent Forms: \textarabic{ي ، يـ ، ـيـ ، ـي}

%%%%%%%%%%%%%%%%%%%%%%%%%%%%%%%%
\subsection{Special cases}

\subsubsection{\textarabic{ء} - hamza}

\subsubsection{\textarabic{ة} - taa' marbuta}

This is the most clever of Arabic characters. It always comes last in
a word (though it may then be suffixed) and may function as either a
(final) soft h (\textarabic{ه}) or as a /t/ (\textarabic{ت}). Its
shape reflects this dual functionality. The name marbuta means
``tied'', and the idea is that the two teeth of a free-standing
\textarabic{ت} have been bound together, resulting in the shape of
\textarabic{ه} with the two dots pushed above it. But this name was a
later invention; Sibawayhi always called this either the ha or that ta
of feminization, since it always marks the word as feminine.

When it is suffixed, it resumes the shape of \textarabic{ت}; in other
words, its free-standing and final forms are the same as
\textarabic{ه}, i.e. \textarabic{ة ، ـة}; but its initial and medial
forms are the same as for \textarabic{ت}.

Only pronounced /h/ when utterance-final, i.e. pausal form. Otherwise
pronounced /t/. Three cases: the word is nunated, suffixed by a
postclitic, or suffixed by a following majrur.

Examples:

\subsubsection{\textarabic{ال} - alif-lam}

Some authors treat this as a first-class \harf.


%%%%%%%%%%%%%%%%%%%%%%%%%%%%%%%%
\subsection{Diacritics - \textit{Taʃkīl}}

\subsubsection{The \textit{ḥarakāt} - voweling}

\subsubsection{shadda}

\subsubsection{Quranic diacritics}

%%%%%%%%%%%%%%%%%%%%%%%%%%%%%%%%
\subsection{Pronunciation Tips}

The following are tricky for English speakers: \textarabic{ص ، ض ، ط ،
  ظ ، ع ، غ ، ق}


%%%%%%%%%%%%%%%%%%%%%%%%%%%%%%%%
\section{Ortho-phonotactics}

\subsection{Waṣl - phonetic continuity}

\subsection{Consonant clusters}

Rule against \textarabic{الْتِقاءُ السَّكِنَيْنِ}


\subsection{idgham}

%%%%%%%%%%%%%%%%%%%%%%%%%%%%%%%%
\section{Orthography}

\subsection{Alif al-Wasl}

E.g. alif is never mutaharrik, but alif wasl may host a damma or
kasra. This does not mean the alif is mutaharrik; rather the alif is
null, and the vowel mark just indicates the lead-in vowel in
pronunciation. For example \textarabic{اِضْرِبْ} strike! Strictly speaking
this is a misspelling, since it is a fundamental principle of Arabic
that speech may not begin with a vowel. So in actual speech this would
be pronounced with a hamza, \textarabic{إِضْرِبْ}. But using a hamza in
the orthography would lead to ambiguity, since normally this would
imply that the alif seat of the hamza is not the “null” alif. So the
correct spelling would be \textarabic{ا}, not \textarabic{إ} or
\textarabic{أ}, and it is up to the reader to infer the implicit
hamza.

For example, \textarabic{اُقتل} unambiguously indicates a command,
“kill!” (\textarabic{اُقْتُلْ}). But spelled with hamza \textarabic{أُقتل}
would normally be read as first person singular passive conjugation,
“I am killed” (\textarabic{اُقْتَلُ}).

Furthermore: the vowel in such cases may be written, but strictly
speaking should be omitted, because the voweling is provided by the
preceding term in continuous speech, which means it depends on
context. The alif is just there to indicate \textit{waṣl} or
continuous ligation. For example: \textarabic{قالَ اقْتَلُ} is articulated
“qâlaqtul”, not “qâla uqtul” (so \textarabic{قالَ اُقْتَلُ} would be a
misspelling).

This is very common with the alif-lam (definite article), whose alif
is alif-wasl, which should never be spelled \textarabic{اَل}. For
example \textarabic{فِي الْبَيْت} is pronounced “filbayt” (short “i”), not
“fīlbayt” nor “fī albayt” nor “fi albayt”. In this case both the long
\textarabic{ي} in \textarabic{فِي} and the alif in \textarabic{الْبَيْت}
are dropped, so the phonetic reading is \textarabic{فِلْبَيْت}.

\subsection{The \textit{tâ'} of Feminization}

Sometimes spelled \textarabic{ة}, and sometimes \textarabic{ت} as in
\textarabic{صفوت} (? find better example)

\subsection{Tanwīn (aka Nunation)}

\subsection{Final Alif with Fatḥatan}

A very widespread misconception is that word-final \textit{fatḥatan}
(\ \textarabic{ـً}\ ) comes after word-final alif, e.g.
\textarabic{ضَرْباً}; in fact this practice is so widespread in modern
publications that it has more or less become the norm. Nevertheless it
is clearly incorrect. The alif is never the “seat” of a vowel mark;
the correct spelling is to place the alif after the \textit{fatḥatan}:
\textarabic{ضَرٍبًا}. The reason for this should be obvious: the /a/
vowel indicated by the \textit{fatḥatan} is the vowel of the final
“term of i'rab” -- in this example, the \textarabic{ب}.

Final alif following \textit{fatḥatan} is phonetically null; rather it
simply serves to alert the reader of unvowelled text that the word
implicitly has accusative case (the /a/ inflection) with nunation:
from \textarabic{ضربا}, the fluent reader can easily infer
\textarabic{ضربًا}.

\subsection{Quranic Alif}

Omission of alif in early orthography, preserved in the mashaf.

Words like \textarabic{صلوات}, spelled as \textarabic{صلوت} or \textarabic{صلوٰت}

%%%%%%%%%%%%%%%%%%%%%%%%%%%%%%%%
\section{Numerical Notations}

\subsection{Traditional}

The traditional abjad ordering of the letters was tied to numbering,
as follows:

\begin{itemize}
\item Units 1-9:\quad \textarabic{أ ب ج د ه‍ و ز ح ط}
\item Tens:\quad \textarabic{ي ك ل م ن س ع ف ص}
\item Hundreds:\quad \textarabic{ي ك ل م ن س ع ف ص}
\item One thousand:\quad \textarabic{غ}
\end{itemize}

\subsection{Decimal}

%%%%%%%%%%%%%%%%%%%%%%%%%%%%%%%%
\section{Arabic at the Keyboard}

\subsection{Unicode}

\href{https://en.wikipedia.org/wiki/Arabic_script_in_Unicode}{Unicode}
contains many “codepoints” used in Arabic orthography across the many
languages that use the Arabic script. Here are a few that are useful
for typesetting Arabic:

\begin{itemize}
\item Arabic Tatweel:\quad\textarabic{ـ}\quad (U+0640). Compare: \textarabic{فتح} and
  \textarabic{فـتـح}
\item Arabic Letter Alef With Madda Above:\quad \textarabic{آ}\quad (U+622)
\item Arabic Letter Alef Wasla:\quad \textarabic{ٱ}\quad (U+0671)
\item Arabic Letter Superscript Alef:\quad \textarabic{ـٰ}\quad (U+0670) e.g. \textarabic{هٰذا}
\item Arabic Letter Dotless Beh:\quad \textarabic{ٮ}\quad (U+066E)
\item Arabic Letter Dotless Qaf:\quad \textarabic{ٯ}\quad (U+066F)
\item Arabic Comma:\quad \textarabic{،}\quad (U+060C) (Also used as
  decimal point)
\item Arabic Semicolon:\quad \textarabic{؛}\quad (U+061B)
\item Arabic Question Mark:\quad \textarabic{؟}\quad (U+061F)
\item Arabic Ornate Left Parenthesis:\quad \textarabic{﴾}\quad (U+FD3E)
\item Arabic Ornate Right Parenthesis:\quad \textarabic{﴿}\quad (U+FD3F)
\end{itemize}

\noindent WARNING: not all fonts support all Arabic codepoints.

\vspace{8pt}
\noindent Forcing or preventing join forms:

\begin{itemize}
\item Zero-width joiner (ZWJ):\quad U+200D - non-printing char that forces
  adjacent characters to print in join form, e.g. to force the initial
  form of \textarabic{ه}, follow it with a ZWJ character:\quad
  \textarabic{ه‍}
\item Zero-width non-joiner (ZWNJ):\quad U+200C - prevents joining
  between adjacent characters, e.g. \textarabic{فح} with ZWNJ:\ \ 
  \textarabic{ف‌ح}. Using a printing space character: \textarabic{ف ح}
\end{itemize}

\vspace{8pt}
\noindent Isolated forms for tashkeel marks: most editors will be
confused if you try to use a “combining” form without a preceding
letter; for example if you try to display a free-standing fatha.
Unicode does contain codepoints for isolated forms, but they are not
well-supported at this time. To display an isolated tashkeel form, put
it on a tatweel, e.g.\quad\textarabic{ـَ} \quad\textarabic{ـِ}\quad etc.

See \href{https://www.w3.org/International/questions/qa-bidi-unicode-controls}{How to use Unicode controls for bidi text}

\subsection{Fonts}



%%%%%%%%%%%%%%%%%%%%%%%%%%%%%%%%
\appendix

\section{License}

Reading Arabic © 2024 by Gregg Reynolds is licensed under Creative
Commons Attribution-NonCommercial-NoDerivatives 4.0 International. To
view a copy of this license, visit
\href{https://creativecommons.org/licenses/by-nc-nd/4.0/}{https://creativecommons.org/licenses/by-nc-nd/4.0/}

\end{document}

