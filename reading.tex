\documentclass[11pt]{article}



% \usepackage[xetex,svgnames,x11names]{xcolor}
%%%%%%%%%%%%%%%%  START XETEX SETUP  %%%%%%%%%%%%%%%%
\usepackage{fontspec}
%%\usepackage{xunicode}
\usepackage{xltxtra}
% \defaultfontfeatures{Mapping=tex-text}

%\setmainfont{Cochin}
\setmainfont{Gentium Book Plus}

% \newfontfamily\symfam{Apple Symbols}
%%\newfontfamily\symfam2{Arial Unicode MS}
%%\newfontfamily\symfam2{/System/Library/Fonts/ZapfDingbats.ttf}
%%\newfontfamily\symfam2{Menlo}

% \newfontfamily\arfont{Scheherazade:script=arab}
% \newfontfamily\arsegfont[Scale=1.2]{Scheherazade:script=arab}
% \newfontfamily\arversefont[Scale=1.5]{Scheherazade:script=arab}

\newfontfamily\englishfontit{Gentium Book Plus Italic}
\newfontfamily\arabicfont[Scale=2.0]{Scheherazade:script=arab}

% \newfontfamily\artitle[Scale=10.0]{Scheherazade:script=arab}
% \newfontfamily\arsubtitle[Scale=3.0]{Scheherazade:script=arab}
% \newfontfamily\arhwfont[Scale=1]{KufiStandardGK:script=arab}
% \newfontfamily\enhwfont[Scale=1.2]{Lucida Grande}

% \newfontfamily\arexfont[Scale=1.2]{Scheherazade:script=arab}

% \newcommand{\exar}[1]{\RLE{\arexfont #1}}

% \newcommand{\ar}[1]{{\beginR\arfont #1\endR}}
% \newcommand{\arnote}[1]{{\RL{\normalsize\arfont #1}}}
% \newcommand{\arheadword}[1]{\beginR{\arhwfont #1}\endR}
% \newcommand{\arx}[2]{\xlit{\itshape #1} ({\beginR\arfont #2\endR})}

% \newcommand{\en}[1]{{\beginL\sffamily #1\endL}}
% \newcommand{\enheadword}[1]{\beginL{\enhwfont #1}\endL}
% \newcommand{\enx}[1]{{\xlit{\itshape #1}}}

% \newfontfamily\entext[Scale=.8]{Geneva}

% \newfontfamily\xlit{Charis SIL}

% \newfontfamily\arxen{Charis SIL}

% % \newfontfamily\roman{Gentium Book Plus}
% \newfontfamily\italic{Gentium Book Plus Italic}
% \newfontfamily\ipafamily{Doulos SIL}
% \newfontfamily\ital{Andale Mono}
% \newfontfamily\symbol[Scale=.8]{Apple Symbols}
\let\XeTeX\undefined
\let\XeLaTeX\undefined
%%%%%%%%%%%%%%%%  END XETEX SETUP  %%%%%%%%%%%%%%%%
\usepackage[pdfusetitle]{hyperref} % Creates hyperlinks and index in the PDF document, preferably load after biblatex
\hypersetup{
    bookmarks=true,         % show bookmarks bar?
    unicode=true,          % non-Latin characters in Acrobat’s bookmarks
    pdftoolbar=true,        % show Acrobat’s toolbar?
    pdfmenubar=true,        % show Acrobat’s menu?
    pdffitwindow=false,     % window fit to page when opened
    pdfstartview={FitH},    % fits the width of the page to the window
    pdftitle={My title},    % title
    pdfauthor={Author},     % author
    pdfsubject={Subject},   % subject of the document
    pdfcreator={Creator},   % creator of the document
    pdfproducer={Producer}, % producer of the document
    pdfkeywords={keyword1} {key2} {key3}, % list of keywords
    pdfnewwindow=true,      % links in new window
    colorlinks=true,        % false: boxed links; true: colored links
    linkcolor=blue,         % color of internal links
    citecolor=green,        % color of links to bibliography
    filecolor=magenta,      % color of file links
    urlcolor=cyan           % color of external links
}

\usepackage{polyglossia}
\setmainlanguage{english}
\setotherlanguage{arabic}

%% \usepackage[extrafootnotefeatures]{bidi}  %% for xetex, load last

\newcommand{\sib}{Sībawayhi}
\newcommand{\ism}{\textit{ism}}
\newcommand{\harf}{\textit{ḥarf}}
\newcommand{\huruf}{\textit{ḥurūf}}
\newcommand{\fil}{\textit{fiʿl}}
\newcommand{\maana}{\textit{maʕnā}}
\newcommand{\damma}{\textit{ḍamma}}
\newcommand{\fatha}{\textit{fatḥa}}
\newcommand{\kasra}{\textit{kasra}}
\newcommand{\sukun}{\textit{sukūn}}
\newcommand{\rafu}{\textit{rafʿ}}
\newcommand{\nasb}{\textit{naṣb}}
\newcommand{\jarr}{\textit{jarr}}
\newcommand{\waqf}{\textit{waqf}}
\newcommand{\tanwin}{\textit{tanwīn}}
\newcommand{\hasan}{\textit{ḥasan}}
\newcommand{\qabih}{\textit{qabīḥ}}
\newcommand{\kitab}{\textit{Kitāb}}
\newcommand{\xabar}{\textit{xabar}}
\newcommand{\maful}{\textit{mafʿūl}}
\newcommand{\zarf}{\textit{ẓarf}}
\newcommand{\qawl}{\textit{qawl}}
\newcommand{\aqwal}{\textit{aqwāl}}
\newcommand{\sama}{\textit{samāʕ}}
\newcommand{\qiyas}{\textit{qiyās}}
\newcommand{\illa}{\textit{ʕilla}}
\newcommand{\taqdir}{\textit{taqdīr}}
\newcommand{\amal}{\textit{ʕamal}}
\newcommand{\hadathan}{\textit{ḥadaθān}}
\newcommand{\hal}{\textit{ḥāl}}
\newcommand{\asl}{\textit{aṣl}}
\newcommand{\almutakallim}{\textit{al-mutakallim}}
\newcommand{\almuxatab}{\textit{al-muxātab}}
\newcommand{\kalam}{\textit{kalām}}
\newcommand{\nahwiyyun}{\textit{naḥwiyyūn}}
%\newcommand{}{\textit{}}


%% ʿ

%%%%%%%%%%%%%%%%%%%%%%%%%%%%%%%%%%%%%%%%%%%%%%%%%%%%%%%%%%%%%%%%
%%%%%%%%%%%%%%%%%%%%%%%%%%%%%%%%%%%%%%%%%%%%%%%%%%%%%%%%%%%%%%%%
\begin{document}

\title{Reading Arabic}
\author{G. A. Reynolds}
\date{Today}
\maketitle

\begin{abstract}
A Tutorial and Guide to Arabic Orthography
\end{abstract}

\tableofcontents

\section{Overview: \textit{ḥarf}, \textit{ḥarakah}, and \textit{taʃkīl}}

Arabic \textit{ḥurûf} have two or four forms, but fortunately each has
a kernel that is more or less the same in each form.

\begin{itemize}
\item freestanding, initial, medial, and final; e.g. \textarabic{ت ، تـ
  ، ـتـ ، ـت}
\item freestanding and final. These have roughly similar shapes. The
  final forms are identical to the freestanding forms, except they are
  tied to the previous form by a ligature:
  \textarabic{ا ـا ، د ـد ، ذ ـذ ، ر ـر ، ز ـز ، و ـو}
\end{itemize}

Note that these forms are determined by the preceding and following
letters rather than position in a word. For example, a word may be
composed entirely of free-standing forms, as in \textarabic{دَوَّرَ}
``dawwara'' or \textarabic{وَرْد} ``ward''; entirely of connected forms,
as in \textarabic{كتب ، منطلق}, or a combination of both, as in
\textarabic{موضع ، كتاب ، دخل ن ضاربون}

Groups:

\begin{itemize}
\item toothed forms: \textarabic{ب ، ت ، ث ، ي}
\item three forms without or with a dot, above or below: \textarabic{ج~،~ح~،~خ}
\item paired forms, without and with a diacritic:
  \textarabic{(د ، ذ) (ر ، ز) (س~،~ش) (ص ، ض) (ط ، ظ) (ع ، غ) (ف ، ق)}
\item remaining forms: \textarabic{ا ، ك ، ل ، م ، و}
\end{itemize}

\section{Ḥurûf (\textarabic{الْحُرُوْف})}

Forms are listed in right-to-left order: freestanding, initial,
medial, final.

Transliterations are for this document only; they deviate somewhat
from standard transliteration schemes.

%%%%%%%%%%%%%%%%%%%%%%%%%%%%%%%%
\subsection{ \textarabic{ا}  \textit{ā} as in b\textit{ā}t}
This is the most complex ḥarf. See below.

\noindent Name: alif

\noindent Forms: \textarabic{ا ، ـا}


%%%%%%%%%%%%%%%%%%%%%%%%%%%%%%%%
\subsection{ \textarabic{ب}  \textit{b} as in \textit{b}ottom}
Mnemonic: \textit{b}ottom dot

\noindent Name: baa'

\noindent Forms: \textarabic{ب ، بـ ، ـبـ ، ـب}


%%%%%%%%%%%%%%%%%%%%%%%%%%%%%%%%
\subsection{ \textarabic{ت} \textit{t} as in \textit{t}wo}
Mnemonic: \textit{t}wo dots on a tooth.

\noindent Name: taa'

\noindent forms: \textarabic{ت ، تـ ، ـتـ ، ـت}

\subsection{ \textarabic{ث} \textit{th} as in \textit{th}ree}
Mnemonic:  \textit{th}ree dots on a tooth.

\noindent Name: taa'

\noindent Transliteration: θ (Greek small letter theta)

\noindent Forms: \textarabic{ث ، ثـ ، ـثـ ، ـث}

\subsection{ \textarabic{ج} \textit{j} as in \textit{j}oker}
Mnemonic:  ?

\noindent Name: jīm (jeem)

\noindent Forms: \textarabic{ج ، جـ ، ـجـ ، ـج}

\subsection{ \textarabic{ح} \textit{ḥ} as in \textit{ḥ}arsh}
Pronunciation: like the ``soft'' h \textarabic{ه}, but much harsher.
See section X below.

\noindent Name: ḥaa

\noindent Transliteration: ḥ (h with dot below)

\noindent Mnemonic:  ?

\noindent Forms: \textarabic{ح ، حـ ، ـحـ ، ـح}

%%%%%%%%%%%%%%%%%%%%%%%%%%%%%%%%
\subsection{ \textarabic{خ} \textit{kh} as in lo\textit{kh} ness}
Pronunciation: like ``\textit{ch}'' in Scottish ``lo\textit{ch}'' or
German ``a\textit{ch}'' but stronger; see below.

\noindent Name: xaa'

\noindent Transliteration: x - ``kh'' is unsuitable, since both
\textit{k} and \textit{h} are used, for \textarabic{ك} and
textarabic{ه} respectively.

\noindent Mnemonic:  ?

\noindent Forms: \textarabic{خ ، خـ ، ـخـ ، ـخ}

%%%%%%%%%%%%%%%%%%%%%%%%%%%%%%%%
\subsection{ \textarabic{د} \textit{d} as in \textit{d}og}
Mnemonic:  ?

This is our first letter with only two forms, freestanding and final.

\noindent Name: dal

\noindent Forms: \textarabic{د ، ـد}

\subsection{ \textarabic{ذ} \textit{th} as in mo\textit{th}er, fa\textit{th}er}
Mnemonic:  a dog with a dot is ...?

\noindent Forms: \textarabic{ذ ، ـذ}

%%%%%%%%%%%%%%%%%%%%%%%%%%%%%%%%
\subsection{ \textarabic{ر} \textit{r} as in \textit{r}ada\textit{r}}
Mnemonic:  ?

\noindent Name: raa'

\noindent Forms: \textarabic{ر ، ـر}

%%%%%%%%%%%%%%%%%%%%%%%%%%%%%%%%
\subsection{ \textarabic{ز} \textit{z} as in \textit{z}ebra}
Mnemonic:  ?

\noindent Name: zayn

\noindent Forms: \textarabic{ز ، ـز}

%%%%%%%%%%%%%%%%%%%%%%%%%%%%%%%%
\subsection{ \textarabic{س} ``thin'' \textit{s} as in \textit{s}ee}
Mnemonic:  ?

\noindent Name: sīn (seen)

\noindent Forms: \textarabic{س ، سـ ، ـسـ ، ـس}

%%%%%%%%%%%%%%%%%%%%%%%%%%%%%%%%
\subsection{ \textarabic{ش} \textit{sh} as in \textit{sh}ow}
Mnemonic:  three dots on three teeth is \textit{sh}owing off

\noindent Name: shīn (sheen)

\noindent Transliteration: ʃ (latin small letter esh)

\noindent Forms: \textarabic{ش ، شـ ، ـشـ ، ـش}

%%%%%%%%%%%%%%%%%%%%%%%%%%%%%%%%
\subsection{ \textarabic{ص} ``thick'' \textit{ṣ}}
Pronunciation: see below.

\noindent Name: ṣaud

\noindent Transliteration: ṣ

\noindent Mnemonic:  ?

\noindent Forms: \textarabic{ص ، صـ ، ـصـ ، ـص}

%%%%%%%%%%%%%%%%%%%%%%%%%%%%%%%%
\subsection{ \textarabic{ض} ``thick'' \textit{ḍ}}
Pronunciation: similar to \textarabic{د}; see below.

\noindent Name: ḍaud

\noindent Transliteration: ḍ (d with dot below)

\noindent Mnemonic:  ?

\noindent Forms: \textarabic{ض ، ضـ ، ـضـ ، ـض}

%%%%%%%%%%%%%%%%%%%%%%%%%%%%%%%%
\subsection{ \textarabic{ط} ``thick'' \textit{ṭ}}
Pronunciation: similar to \textarabic{د}; see below.

\noindent Name: ṭaa

\noindent Transliteration: ṭ (t with dot below)

\noindent Mnemonic:  ?

\noindent Forms: \textarabic{ط ، طـ ، ـطـ ، ـط}

%%%%%%%%%%%%%%%%%%%%%%%%%%%%%%%%
\subsection{ \textarabic{ظ} ``thick'' \textit{ẓ}}
Pronunciation: similar to \textarabic{ذ}, \textarabic{ز}; see below.

\noindent Name: ẓaa

\noindent Transliteration: ẓ (z with dot below)

\noindent Mnemonic:  ?

\noindent Forms: \textarabic{ظ ، ظـ ، ـظـ ، ـظ}

%%%%%%%%%%%%%%%%%%%%%%%%%%%%%%%%
\subsection{ \textarabic{ع} (unpronouncable ;)}
Pronunciation: see below

\noindent Name: 'ayn

\noindent Transliteration: 9 (rhymes with ayn, vaguely similar shape)

\noindent Mnemonic:  ?

\noindent Forms: \textarabic{ع ، عـ ، ـعـ ، ـع}

%%%%%%%%%%%%%%%%%%%%%%%%%%%%%%%%
\subsection{ \textarabic{غ} \textit{g} as in \textit{g}ar\textit{g}le}
Pronunciation: a guttural, rolling, hard /g/ sound

\noindent Name: ghayn

\noindent Transliteration: g

\noindent Mnemonic:  ?

\noindent Forms: \textarabic{غ ، غـ ، ـغـ ، ـغ}

%%%%%%%%%%%%%%%%%%%%%%%%%%%%%%%%
\subsection{ \textarabic{ف} \textit{f} as in \textit{f}face}

\noindent Name: faa'

\noindent Transliteration: f

\noindent Mnemonic:  ?

\noindent Forms: \textarabic{ف ، فـ ، ـفـ ، ـف}

%%%%%%%%%%%%%%%%%%%%%%%%%%%%%%%%
\subsection{ \textarabic{ق} \textit{q} as in \textit{q}ueue}
Pronunciation: like /k/ but farther back in the throat; see below

\noindent Name: qaf

\noindent Transliteration: q

\noindent Mnemonic:  ?

\noindent Forms: \textarabic{ق ، قـ ، ـقـ ، ـق}

%%%%%%%%%%%%%%%%%%%%%%%%%%%%%%%%
\subsection{ \textarabic{ك} \textit{k} as in \textit{k}ick}

\noindent Name: kaf

\noindent Transliteration: k

\noindent Mnemonic:  ?

\noindent Forms: \textarabic{ك ، كـ ، ـكـ ، ـك}

%%%%%%%%%%%%%%%%%%%%%%%%%%%%%%%%
\subsection{ \textarabic{ل} \textit{l} as in \textit{l}amb}

\noindent Name: lam

\noindent Transliteration: l

\noindent Mnemonic:  ?

\noindent Forms: \textarabic{ل ، لـ ، ـلـ ، ـل}

%%%%%%%%%%%%%%%%%%%%%%%%%%%%%%%%
\subsection{ \textarabic{م} \textit{m} as in \textit{m}i\textit{m}e}

\noindent Name: mīm (meem)

\noindent Transliteration: m

\noindent Mnemonic:  ?

\noindent Forms: \textarabic{م ، مـ ، ـمـ ، ـم}

%%%%%%%%%%%%%%%%%%%%%%%%%%%%%%%%
\subsection{ \textarabic{ن}  \textit{n} as in \textit{n}oo\textit{n}, o\textit{n}e}

\noindent Name: nûn (noon)

Mnemonic:  o\textit{n}e dot on a tooth.

NB: this is a “tooth” letter, like \textarabic{ب ، ت ، ي}, but the
freestanding and final forms are a little deeper.

\noindent Forms: \textarabic{ن ، نـ ، ـنـ ، ـن}

%%%%%%%%%%%%%%%%%%%%%%%%%%%%%%%%
\subsection{ \textarabic{ه} soft \textit{h} as in  \textit{h}um, \textit{h}am, \textit{h}im }

\noindent Name: haa'

\noindent Transliteration: h

\noindent Mnemonic:  ?

\noindent Forms: \textarabic{ه ، هـ ، ـهـ ، ـه}

%%%%%%%%%%%%%%%%%%%%%%%%%%%%%%%%
\subsection{ \textarabic{و} \textit{w} as in  \textit{w}ater,
  \textit{oo} as in b\textit{oo}t}

\noindent Name: waw

\noindent Transliteration: w

\noindent Mnemonic:  ?

\noindent Forms: \textarabic{و ، ـو}

%%%%%%%%%%%%%%%%%%%%%%%%%%%%%%%%
\subsection{\textarabic{ي}  \textit{y} as in \textit{y}et;
  \textit{ee} as in f\textit{ee}t; \textit{y} as in fish\textit{y}}
Mnemonic:  ?

\noindent Name: yaa'

\noindent Transliteration: y

\noindent Mnemonic:  ?

\noindent Forms: \textarabic{ي ، يـ ، ـيـ ، ـي}


%%%%%%%%%%%%%%%%%%%%%%%%%%%%%%%%
\subsection{Pronunciation Tips}

The following are tricky for English speakers: \textarabic{ص ، ض ، ط ،
  ظ ، ع ، غ ، ق}

%%%%%%%%%%%%%%%%%%%%%%%%%%%%%%%%
\section{Special cases}

\subsection{\textarabic{ى , آ , ٱ , ا} - varieties of \textit{alif}}

\subsubsection{\textarabic{ٱ} alif al-wasl}
\subsubsection{\textarabic{آ} alif al-madd}
\subsubsection{\textarabic{ـٰ}\  dagger alif}
\subsubsection{\textarabic{ى} alif maqsurah}

The strangest beast in the menagerie. The alif maqsura may be written
as either \textarabic{ى}, as in \textarabic{فتى}, or \textarabic{ا} as
in \textarabic{غَزَا}. Which demonstrates that \textit{alif} is a
concept, rather than a mere letter.

See also \textarabic{الممدود} e.g. \textarabic{بِنَاْء}.

WARNING: in printed text, final \textarabic{ي} is often written
without dots, so it looks like a proper alif maqsurah.



\subsection{\textarabic{ء} - hamza}

\subsection{\textarabic{ة} - taa' marbuta}

This is the most clever of Arabic characters. It always comes last in
a word (though it may then be suffixed) and may function as either a
(final) soft h (\textarabic{ه}) or as a /t/ (\textarabic{ت}). Its
shape reflects this dual functionality. The name marbuta means
``tied'', and the idea is that the two teeth of a free-standing
\textarabic{ت} have been bound together, resulting in the shape of
\textarabic{ه} with the two dots pushed above it. But this name was a
later invention; Sibawayhi always called this either the ha or that ta
of feminization, since it always marks the word as feminine.

When it is suffixed, it resumes the shape of \textarabic{ت}; in other
words, its free-standing and final forms are the same as
\textarabic{ه}, i.e. \textarabic{ة ، ـة}; but its initial and medial
forms are the same as for \textarabic{ت}.

Only pronounced /h/ when utterance-final, i.e. pausal form. Otherwise
pronounced /t/. Three cases: the word is nunated, suffixed by a
postclitic, or suffixed by a following majrur.

Examples:

%%%%%%%%%%%%%%%%%%%%%%%%%%%%%%%%
\section{\textit{Taʃkīl}}

\subsection{shadda}

\subsection{idgham}

\subsection{Quranic orthography}

%%%%%%%%%%%%%%%%%%%%%%%%%%%%%%%%
\section{Order}

\subsection{Alif-baa'}

\subsection{Abjadiyyah}

%%%%%%%%%%%%%%%%%%%%%%%%%%%%%%%%
\appendix

\section{License}

Reading Arabic © 2024 by Gregg Reynolds is licensed under Creative
Commons Attribution-NonCommercial-NoDerivatives 4.0 International. To
view a copy of this license, visit
\href{https://creativecommons.org/licenses/by-nc-nd/4.0/}{https://creativecommons.org/licenses/by-nc-nd/4.0/}

\end{document}

